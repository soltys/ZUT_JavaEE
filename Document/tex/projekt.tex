\documentclass{article}
\usepackage[MeX]{polski}
\usepackage{parskip}
\usepackage[utf8]{inputenc}
\title{Portal blogowy}
\author{\textsc{Paweł Sołtysiak I1-41A} \\ \texttt{psoltysiak@wi.zut.edu.pl}}
\begin{document}
\maketitle
\section{Wstęp}
Portal blogowy ma być platformą łącząca twórców blogów. Wspomagających ich komunikację oraz stanowić sposób na wypromowanie własnej twórczości wobec innych użytkowników portalu.



\section{Wymagania funkcjonalny}
Portal musi posiadać możliwość zakładanie kont użytkowników. Użytkowników dzieli się następujące grupy.
\begin{itemize}
\item Normalni -- osoby posiadają możliwość tworzenia blogów, usuwania własnych blogów, posiadają prawo do wprowadzania modyfikacji w własnych blogach. Posiadają możliwość wstawiania komentarzy w blogu.
\item Administratorzy -- mają pełne prawa w portalu, mogą tworzyć własne blogi mogę usuwać blogi innych osób, nie mogą edytować zawartości innych blogów.
\end{itemize}

Portalu musi posiadać funkcjonalność zakładania własnych blogów. Każdy blog jest przypisany do jednej konkretnej osoby. Właściciel bloga może z własnego bloga dodawać nowe treści, edytować lub je usuwać.

Każdy z użytkowników przypisanych do grupy Normalnej posiada możliwość zostawiana komentarzy pod wpisami wprowadzonymi wewnątrz bloga.

\section{Wymagania niefunkcjonalne}
\section{Diagram przypadków użycia}
%next-next exam
\section{Diagram klas}
\section{Diagram komponentów}

\end{document}

